\section{Introduction}

  \indent Personalization is no more a novelty. The first emergence of this phenomenon dated back to antiquity
  when experienced merchants provided different customers with different products or services \cite{adomavicius2008personalization}. However, the real interest of personalization did not arise until the late 1990s with the advancement of the Web technologies \cite{don2011managing}.
  From then on, more studies have been done in this field, and people try to explore the essence of personalization.

  \indent In the world of personalization, one thing we might often hear about is recommender systems (RS), which intelligently give users personalized suggestions. The recommendations can vary from algorithm to algorithm. The most common forms are the content-based recommendation, and the collaborative filtering recommendation. The first one, content-based recommendation, observes utility of previous items and then use this attribute to predict the utility of other items \cite{balabanovic1997fab}. An example can be ``User A has bought a novel. And based on this bought history, a magazine is more likely to be recommended to him than a pair of shoes, because novels and magazines are more in common than novels and shoes ''. The second type, collaborative filtering, observes utility for user-item pairs which is then used to predict utility for unobserved user-item pairs \cite{balabanovic1997fab}. One example can be ``User A(1,2) has bought items 1 and 2, user B(1,2,3) has purchased items 1,2 and 3 and user C(2,4) has bought items 2 and 4. As a recommendation for user A, item 3 is more likely to be suggested to user A than item 4, since user A and user B are more similar than user A and user C''.

  \indent A good recommendation algorithm is no doubt indispensable for a recommender system, and the way of explaining its complex algorithm to users
  plays also an important role. With the development of the Internet, the privacy issue becomes more and more important. The newly EU regulation even requires explanations for complex algorithms \cite{goodman2016european}. People are willing to know which information is ``consumed'' by the system and they may lose their trust in one system if it behaves differently as they expected. 
  In this case, a proper explanation of a recommender system can help to inspire users' trust and loyalty, increase satisfaction, make it quicker and easier for users to find what they want \cite{tintarev2007survey}. In order to answer the question, "What makes a good explanation," we check through different studies and try to summarize a set criteria in this paper. Besides, we particularly focus on brief explanations.

  \indent The rest of the paper is divided into four parts. The first section provides an overview of different studies on explanations in recommender systems. The second part focuses on two frameworks, which are often used by other researchers to evaluate and design explanations of recommender systems. The third section extracts some principles from the two frameworks, which can be used to design ``brief explanations'' in an automotive scenario. And in the last part, we summarize our contribution and propose several questions for future studies.

\section{Literature Review}

    \indent In this section, we try to give readers an overview of state of the art. We introduce in this part some relevant studies which focus on the recommendation and its explanations.

    \indent User experience research is increasingly attracting researchers’ attention. Some researchers, like Li Chen \cite{pu2011user}, and Bart P.Knijnenburg \cite{knijnenburg2012explaining}, put emphasis on recommender system and try to figure out what constitutes an effective and satisfying recommender system. Other researchers focus on its explanations. They have proved that a good explanation can boost users' trust towards the system and help them in decision making process \cite{tintarev2007survey} \cite{van2004designing} \cite{pu2007trust}.
    
    \indent One possible way to evaluate explanations of a recommender system is to look at the soundness and completeness \cite{kulesza2013too}. The soundness means the extent to which an explanation describes all of the underlying systems. Meanwhile, the completeness means how truthful each element in an explanation is, in the perspective of a user, in the underlying system. The author has found out that LSHC(which means low soundness and high completeness) provides participants with the best-perceived cost/benefit ratio of attending to the explanations. 

    \indent Another possible method to evaluate the explanations is to categorize explanations in different types of questions. In Lim's work, Toolkit Support Intelligibility in Context-Aware Applications \cite{lim2010toolkit}, they divided the explanation of a context-aware system into eight questions, which are Why, Why Not, How To, What, What If, Inputs, Outputs, and Certainty. In the findings, they emphasized on streamlining explanations while maintaining access to the rich explanation capabilities, which makes explanations usable and quickly consumable.    
    
    \indent Finally, a significant breakthrough is made by Nava Tintarev and Judith Masthoff. They have proposed what they called ``seven possible advantages of an explanation facility \cite{tintarev2007survey}.'' These seven advantages are often taken as a standard criteria in evaluating and designing explanations for a recommender system.