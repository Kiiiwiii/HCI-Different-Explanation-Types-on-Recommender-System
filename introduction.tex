\section{Introduction}

  \indent Personalization has already existed for a long time. The first emergence of this phenomena 
  can be traced back to antiquity,
  when experienced merchants provided different customers with different products or services\cite{adomavicius2008personalization}. However, 
  the real interest of personalization arose in the mid to late 1990s with the advancement of the Web technologies\cite{don2011managing}.
  From then on, more efforts have been investigated in this field in order to decode the essence of personalization.
  
  \indent In the world of personalization, the most common form of it is recommender systems (RS), which can intelligently
  give users personalized suggestions. By adopting different algorithms, the recommender systems can also be different.
  Content-based recommendation and collaborative filtering recommendation are two popular types of recommendation, on which large amount 
  of researches have been done. The first one, content-based recommendation, observes utility of items experienced previously and
  their attributes for a given user is taken as input to predict the utility of other items to that user\cite{balabanovic1997fab}.
  A simpler explanation can be ``User A has bought a novel. And based on this history, a magazine is more likely to be recommended to him than a pair of shoes,
  because novels and magazines are more in common than novels and shoes ''. The second type, collaborative filtering, observes utility for user-item pairs is taken
  as input to predict utility for unobserved user-item pairs\cite{balabanovic1997fab}. One example can be ``User A(1,2) has bought items 1 and 2, user B(1,2,3) has bought items 1,2 and 3 and user C(2,4) has bought items 2 and 4.
  Thus, item 3 is more likely to be recommended to user A than item 4, because user A and user B are more similar than user A and user C''.
  
  \indent A good recommendation algorithm is no doubt indispensable for a recommender system and the way of explaining its complex algorithm to users
  plays also an important role. With the development of the Internet, the privacy issue gains more and more importance in the society
  and people are willing to know which information is ``consumed'' by the system. They may lose their trust on the system if
  it behaves another way as they expected. In this case, a good explanation of the behaviors of a recommender system 
  can help inspire user trust and loyalty, increase satisfaction, make it quicker and easier for users to find what they want\cite{tintarev2007survey}.

  The following content are divided into four parts. The first part provided an overview of different researches on explanations in recommender systems. 
  The second part focused on two possible methods or standards, which are recently often used by researches to evaluate and design explanations of recommender systems.
  while the third extracts some factors from these two methods, which can be used to design ``short explanations'' in automotive scenario.
  And the last part discussed on some limitations in recent researches, proposed several research questions for future research and provided
  a summary of this whole paper.

\section{Overview}

    \indent 
    User experience research is increasingly attracting researchers’ attention in the recommender system community. 
    Some researchers, like Li Chen\cite{pu2011user} and Bart P. Knijnenburg\cite{knijnenburg2012explaining}, focus on a larger scale and try to figure out what constitutes an effective and satisfying recommender system.
    Other researchers focus on a more detailed part, that is the explanation of the system. They have proved that
    a good explanation can boost users' trust towards the system and help them in decision making process\cite{tintarev2007survey}\cite{van2004designing}\cite{pu2007trust}.

    \indent David Mcsherry proposed a case-based reasoning (CBR) approach\cite{mcsherry2005explanation} to product recommendation
    that offers important benefits in terms of the ease with which the recommendation process can be explained and the system\rq s recommendations can be justified.

    \indent Another perspective focused on how the soundness and completeness of the explanations impacts the fidelity of end users' mental models\cite{kulesza2013too},
    where the soundness means the extend to which an explanation describes all of the underlying system
    and the completeness means how truthful each element in an explanation is with respect to the underlying system.

    \indent And Brian Y. Lim and Anind K. Dey dived into question types. In their work, Toolkit Support Intelligibility in Context-Aware Applications\cite{lim2010toolkit}, they
    divided explanation types into eight factors, which are Why, Why Not, How To, What, What If, Inputs, Outputs and Certainty. More details are explained in later section.
    
    \indent Finally, a great breakthrough has been made by Nava Tintarev and Judith Masthoff.
    They have proposed what they called ``seven possible advantages of an explanation facility''\cite{tintarev2007survey}.
    These seven advantages are often taken as a standard criterion to evaluate different types of explanation in other researchers' works.

