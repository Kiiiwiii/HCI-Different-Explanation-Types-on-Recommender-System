\section{Introduction}
\label{chapter: intro}
\subsection{A Brief Introduction of Recommender system}
    \indent Recommender Systems (RSs) are software tools and techniques providing suggestions for items to be of use to a user\cite{ricci2011introduction}.
    A simplest form of them is a ranking list, where users  can find the most popular items
    and in turn adjust their choices (like the ranking list of AppStore). Although it is not
    personalized, it is still a kind of recommendation. 
    
    \indent More complex forms of RS are like
    content-based recommendation, user-based recommendation and recommendation based on 
    collaborative filtering, where users get more specific recommendations based on their 
    previous behaviors, which are, for example, the history views in Youtube or history purchased item in Amazon.
    
    \indent With the development of the Internet, the privacy issue gains more and more importance in the society
    and people are willing to know which information is ``consumed'' by the system. They tend to be doubt if
    the system behaves another way as they expected. In this case, a good explanation of the behaviors of a recommendation system 
    can help inspire user trust and loyalty, increase satisfaction, make it quicker and easier for users to find what they want\cite{tintarev2007survey}.
\subsection{Related Works}
    \indent A large amount of researches have been done in order to find out what makes a good explanation.
    David Mcsherry proposed a case-based reasoning (CBR) approach\cite{mcsherry2005explanation} to product recommendation
    that offers important benefits in terms of the ease with 
    which the recommendation process can be explained and the system\rq s recommendations can be justified.

    \indent Another perspective focused on how the soundness and completeness of the explanations impacts the fidelity of end \text{users\textquotesingle} mental models\cite{kulesza2013too},
    where the soundness means the extend to which an explanation describes all of the underlying system
    and the completeness means how truthful each element in an explanation is with respect to the underlying system.

    \indent Besides, TODO: a brief intro to: Toolkit to Support Intelligibility in Context-Aware Applications\cite{lim2010toolkit}.
    
    \indent TODO: The most popular standard which is used to evaluate the explanation is the explanatory criterion\cite{tintarev2007survey}.