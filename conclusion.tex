\section{Discussion, Future Work and Conclusion}

    \indent With the development of the automobile and mobile industry, more and more recommender systems need to work in a context-aware environment and compared with the desktop environment, users can not fully concentrate on the device because their environment requires more attention at the same time. In such cases, an efficient and brief explanation which informs the system's behaviors is especially important.
    
    \indent However, unlike typical explanations in a recommender system (explanations for desktop recommender systems), which has already some ``standard'' methods to design them. There is no ``standard'' approach to design and evaluate brief explanations. 

    \indent In the first study, mentioned in section 4.1, Roland Bader extracted two criteria from the first framework - Seven Explanatory Criteria for brief explanations. He has successfully proved that these two principles (\textbf{efficiency} and \textbf{persuasiveness}) from the first framework are suitable for brief explanations. However, he did not mention other principles. Thus, it might be helpful if further studies can examine and evaluate other criteria (\textbf{transparency}, \textbf{scrutability}, \textbf{effectiveness}, \textbf{trust} and \textbf{satisfaction}). In the second study,  Martin Steinert only evaluated two types of explanations (\textbf{what} and \textbf{why}) extracted from the second framework, Question-Based-Explanation. Therefore, some further studies can focus on other question-type-based explanations (\textbf{why not}, \textbf{how to}, \textbf{what if}, \textbf{inputs}, \textbf{outputs} and \textbf{certainty}).

    \indent In conclusion, in this paper, we firstly provided a literature overview of explanations in personalization and recommender systems. Secondly, we introduced two frameworks (Seven Explanatory Criteria and Question-Type-Based Explanations) which are used to design and evaluate explanations. Moreover, we illustrated how these two frameworks are applied in Google Advertisement recommender system. Thirdly, we introduced two studies which extracted some principles (\textbf{efficiency}, \textbf{persuasiveness} and \textbf{why} question-type-based explanations) from the two frameworks mentioned above, in order to find a proper approach to design brief explanations for a assistant recommender system in a car. And finally, we summarized our contributions and proposed some possible directions for future studies.