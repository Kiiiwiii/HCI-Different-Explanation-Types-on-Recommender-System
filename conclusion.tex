\section{Discussion and Limitations}

\indent  With the development of automobile industry and mobile markets, more and more recommender systems
need to work in a context-aware environment and compared with the desktop environment, users can 
not fully concentrate on the device because they also need to pay attention to environment at the same time.
In such cases, a clear and brief explanation which informs the system's behaviors to users is especially important.

\indent However, unlike the general explanations of recommender systems (explanations for desktop recommender systems), which has already some ``standard''
methods (seven explanatory criterion or question-based explanations).
There is no ``standard'' approach to design and evaluate short explanations. 

\indent And compared with explanations of a recommender system in desktop scenario, which can contain pages of explanations,
the explanation for automobile scenario is quite limited in the perspective of information amount. Thus, we need to extract some criterion from 
the already existed standards of general explanations and check their validity for short explanations.

\indent In Roland Bader's study, although he has successfully extracted two criterion (\textbf{efficiency} and \textbf{persuasiveness})
for short explanations, he did not explain why he choose these two criterion specifically. 
We still do not know whether some other criterion from the seven explanatory criterion are also suitable for short explanations.
Thus, it might be helpful if further studies can examine and evaluate other criterion (\textbf{transparency}, \textbf{scrutability}, \textbf{effectiveness}, \textbf{trust} and \textbf{satisfaction}) for short explanations.

The same goes with the study of Martin Steinert. Only two types of explanations (\textbf{what} and \textbf{why}) were evaluated.
Therefore, some further studies can focus on other question-type-based explanations (\textbf{why not}, \textbf{how to}, \textbf{what if}, \textbf{inputs}, \textbf{outputs} and \textbf{certainty}).

\section{Conclusion}

\indent In this paper, it firstly provided a literatural overview of explanations in personalization and recommender systems.
Then two kind of standards were introduced in order to 
gain a better understanding of the methods used to design and evaluate explanations.
Thirdly, an example of explanations on Google Advertisement was presented so that readers can understand how 
these two methods/standards look like in reality. 
In the next, we introduced two studies of short explanations in automobile scenario
which extracted some useful information about how to especially design and evaluate short explanations.
In the end, we discussed the limitations of both two studies on short explanations and proposed some possible directions for future studies.