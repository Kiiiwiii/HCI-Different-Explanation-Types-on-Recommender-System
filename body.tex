
\section{Explanatory Criterion and two Types of Explanation}
\subsection{Seven Explanatory Criterion}
    \indent The explanatory criterion\cite{tintarev2007survey} (see table \ref{table:1}), originally proposed by Nava Tintarev and Judith Masthoff, 
    are mostly used when considering to design the explanation of a recommender system. 

    \begin{table}[ht] 
        % ht used to attach the table to the position approximately where they are wrote here
        \centering
        \begin{tabular}{ | m{8em} | m{4cm} |} 
        \hline
        %  \bfseries used to bold the header
         \bfseries Aim & \bfseries Definition\\ [0.5ex] 
        \hline\hline
        Transparency & Explain how the system works\\ 
        \hline
        Scrutability & Allow users to tell the system it is wrong\\ 
        \hline
        Trust & Increase \text{users\textquotesingle} confidence in the system\\ 
        \hline
        Effectiveness & Help users make good decisions\\ 
        \hline
        Persuasiveness & Convince users to try\\ 
        \hline
        Efficiency & Help users make decisions faster\\ 
        \hline
        Satisfaction & Increase the ease of use or enjoyment\\ 
        \hline
        \end{tabular}
        \caption{Explanatory criteria and their definitions}
        \label{table:1}
    \end{table}
    \indent Although they have different names in different researches. For example, 
    Mohammed Z.Taie call them explanation attributes\cite{al2013explanations},
    which represent the benefits explanations provide to recommender systems and Fatih Gedikli call them quality factors, 
    which he used to evaluate different explanation types in his study\cite{gedikli2014should}.
    They have the same purpose, that is, to make the system more understandable by users.
    These criterion are listed here as follows:
    
    \begin{enumerate}
        \item \textbf{Transparency:} TODO:// write the intro
        \item \textbf{Scrutability:} TODO:// write the intro
        \item \textbf{Trust:} TODO:// write the intro
        \item \textbf{Effectiveness:} TODO:// write the intro
        \item \textbf{Persuasiveness:} TODO:// write the intro
        \item \textbf{Efficiency:} TODO:// write the intro
        \item \textbf{Satisfaction:} TODO:// write the intro
    \end{enumerate}

\subsection{Type 1 -``Detailed Explanation''}
    TODO: The definition of Detailed Explanation?
    Most of the recommender systems on desktop scenario adapted the ``Detailed Explanation''.
    They covered almost all of the seven explanatory criterion that mentioned above......TODO: complete the intro part
\subsection{Examples for ``Detailed Explanation''}
    An example of explanation in amazon.com.
    TODO: (Write Analysis based on seven explanatory criterion) 
    \begin{figure}[H]
        \centering
        \includegraphics[width=0.5\textwidth]{figures/amazon1}
        \caption{Amazon}
        \label{fig:amazon1}
    \end{figure}
    An example of explanation of Google Ads.
    TODO:  (Write Analysis based on seven explanatory criterion) 
    \begin{figure}[H]
        \centering
        \includegraphics[width=0.5\textwidth]{figures/google1}
        \caption{Google}
        \label{fig:google1}
    \end{figure}
\subsection{Type 2 -``Brief Explanation''}
    However, in some scenario, a brief explanation is more suitable, for example, driving in a car
    (why: users have limited attention resource)
    We can not take all seven explanatory criterion into consideration.
    How to extract a kind of standards.
\subsection{Examples for ``Brief Explanation''}
    Example1: Explanations in Proactive Recommender Systems in Automotive Scenarios \cite{bader122011explanations}
    \textbf{Extract two criterion out of seven explanatory criterion: Persuasiveness and Efficiency.}

    Example2: Why did my car just do that? Explaining semi-autonomous driving actions to improve driver understanding, trust, and performance\cite{koo2015did}
    \textbf{Extract two types (what and how)based on question types from Intelligent Toolkit} \cite{lim2010toolkit, lim2011design}
\subsection{Comparison between ``Detailed Explanation'' and ``Brief Explanation''}
